%%%%%%%%%%%%%%%%%%%%%%%%%%%%%%%%%%%%%%%%%%%%%%%%%%%%%%%%%%%%%%%%%%%%%%%%%%%%%%%%
% VPIC Gordon Bell submission 2008
%
% $LastChangedRevision$
% $LastChangedDate$
% $LastChangedBy$
%
%%%%%%%%%%%%%%%%%%%%%%%%%%%%%%%%%%%%%%%%%%%%%%%%%%%%%%%%%%%%%%%%%%%%%%%%%%%%%%%%
\documentclass[letter,10pt]{article}

%%%%%%%%%%%%%%%%%%%%%%%%%%%%%%%%%%%%%%%%%%%%%%%%%%%%%%%%%%%%%%%%%%%%%%%%%%%%%%%%
% packages
%%%%%%%%%%%%%%%%%%%%%%%%%%%%%%%%%%%%%%%%%%%%%%%%%%%%%%%%%%%%%%%%%%%%%%%%%%%%%%%%
\usepackage{amsmath}
\usepackage{color}
\usepackage{latexmake}

%%%%%%%%%%%%%%%%%%%%%%%%%%%%%%%%%%%%%%%%%%%%%%%%%%%%%%%%%%%%%%%%%%%%%%%%%%%%%%%%
% notes
%%%%%%%%%%%%%%%%%%%%%%%%%%%%%%%%%%%%%%%%%%%%%%%%%%%%%%%%%%%%%%%%%%%%%%%%%%%%%%%%
\newcommand{\note}[1]{\marginpar{\small{\color{red} #1}}}

%%%%%%%%%%%%%%%%%%%%%%%%%%%%%%%%%%%%%%%%%%%%%%%%%%%%%%%%%%%%%%%%%%%%%%%%%%%%%%%%
% add to document dimensions
% fudge if we need to add more space
%%%%%%%%%%%%%%%%%%%%%%%%%%%%%%%%%%%%%%%%%%%%%%%%%%%%%%%%%%%%%%%%%%%%%%%%%%%%%%%%
%\addtolength{\topmargin}{-2cm}
%\addtolength{\textheight}{3cm}
%\addtolength{\oddsidemargin}{-2.25cm}
%\addtolength{\evensidemargin}{-2.25cm}
%\addtolength{\textwidth}{4.5cm}

%%%%%%%%%%%%%%%%%%%%%%%%%%%%%%%%%%%%%%%%%%%%%%%%%%%%%%%%%%%%%%%%%%%%%%%%%%%%%%%%
% article information
%%%%%%%%%%%%%%%%%%%%%%%%%%%%%%%%%%%%%%%%%%%%%%%%%%%%%%%%%%%%%%%%%%%%%%%%%%%%%%%%
\title{400 TFlop/s Trillion-Particle Particle-in-Cell Modeling of Laser Plasma Interactions on Roadrunner}
\author{%
B. Albright\thanks{Applied Physics Division (X-1-PTA Plasma Theory and Applications), Los Alamos National Laboratory, Los Alamos, NM 87544, Email: \emph{balbright@lanl.gov}} \and%
%
K. Barker\thanks{Computer, Computational, and Statistical Sciences Division (CCS-1 Computer Science on High Performance Computing), Los Alamos National Laboratory, Los Alamos, NM 87544, Email: \emph{kjbarker@lanl.gov}} \and%
%
B. Bergen\thanks{Computer, Computational, and Statistical Sciences Division (CCS-2 Computational Physics), Los Alamos National Laboratory, Los Alamos, NM 87544, Email: \emph{bergen@lanl.gov}} \and%
%
K. Bowers\thanks{D.E. Shaw Research LLC, 120 W 45th Street, 39th Floor, New York, NY 10036, Email: \emph{kevin.j.bowers@gmail.com}} \and%
%
D. Kerbyson\thanks{Computer, Computational, and Statistical Sciences Division (CCS-1 Computer Science on High Performance Computing), Los Alamos National Laboratory, Los Alamos, NM 87544, Email: \emph{djk@lanl.gov}} \and%
%
L. Yin\thanks{Applied Physics Division (X-1-PTA Plasma Theory and Applications), Los Alamos National Laboratory, Los Alamos, NM 87544, Email: \emph{lyin@lanl.gov}}}
\date{\today}

% if we have assessment, then we need to include CCS-1 types, no? 

%%%%%%%%%%%%%%%%%%%%%%%%%%%%%%%%%%%%%%%%%%%%%%%%%%%%%%%%%%%%%%%%%%%%%%%%%%%%%%%%
% begin document
%%%%%%%%%%%%%%%%%%%%%%%%%%%%%%%%%%%%%%%%%%%%%%%%%%%%%%%%%%%%%%%%%%%%%%%%%%%%%%%%
\begin{document}

%%%%%%%%%%%%%%%%%%%%%%%%%%%%%%%%%%%%%%%%%%%%%%%%%%%%%%%%%%%%%%%%%%%%%%%%%%%%%%%%
% title
%%%%%%%%%%%%%%%%%%%%%%%%%%%%%%%%%%%%%%%%%%%%%%%%%%%%%%%%%%%%%%%%%%%%%%%%%%%%%%%%
\maketitle
\thispagestyle{empty}

\note{Author list may be incomplete.}

%%%%%%%%%%%%%%%%%%%%%%%%%%%%%%%%%%%%%%%%%%%%%%%%%%%%%%%%%%%%%%%%%%%%%%%%%%%%%%%%
% abstract
%%%%%%%%%%%%%%%%%%%%%%%%%%%%%%%%%%%%%%%%%%%%%%%%%%%%%%%%%%%%%%%%%%%%%%%%%%%%%%%%
\begin{abstract}
\note{Place holder abstract.  If we keep any of this we might not want to mention Livermore ;-)}
We demonstate the outstanding performance and scalability of the VPIC kinetic plasma modeling code
on the heterogeneous IBM Roadrunner supercomputer at the Los Alamos National Laboratory.  VPIC is a  
three-dimensional, relativistic, electromagnetic, particle-in-cell code that self-consistently evolves 
a kinetic plasma.  VPIC simulations of laser plasma interacion (LPI) have been conducted at unprecedented 
fidelity and scale---one trillion simulation macroparticles on 125 million computational cells--to 
model accurately the plasma environment inside a laser-driven hohlraum in an inertial confinement fusion 
experiment.   Sustained performance of approximately 0.4~Pflop/s was achieved.  This capability opens up
the exciting possibility of using VPIC to model in a first-principles manner a problem that threatens the
success of the multi-billion dollar DOE/NNSA National Ignition Facility.  
%
%In this paper, we demonstrate the unprecedented performance and scalability of VPIC, a three-dimensional, 
%relativistic, kinetic plasma simulation code on the heterogeneous, IBM RoadRunner supercomputer at Los Alamos 
%National Laboratory.  Simulations of Laser-Plasma Interactions (LPI) with up to one trillion particles, 
%which achieve a sustained performance of ~0.4 Pflop/s, are detailed.  This work will enable significant 
%breakthroughs in our understanding of plasma reflectivity in Inertial Confinement Fusion (ICF) at the National 
%Ignition Facility (NIF), located in Livermore, California.
\end{abstract}

%%%%%%%%%%%%%%%%%%%%%%%%%%%%%%%%%%%%%%%%%%%%%%%%%%%%%%%%%%%%%%%%%%%%%%%%%%%%%%%%
% add break after title page
%%%%%%%%%%%%%%%%%%%%%%%%%%%%%%%%%%%%%%%%%%%%%%%%%%%%%%%%%%%%%%%%%%%%%%%%%%%%%%%%
\pagebreak

%%%%%%%%%%%%%%%%%%%%%%%%%%%%%%%%%%%%%%%%%%%%%%%%%%%%%%%%%%%%%%%%%%%%%%%%%%%%%%%%
% introduction
%%%%%%%%%%%%%%%%%%%%%%%%%%%%%%%%%%%%%%%%%%%%%%%%%%%%%%%%%%%%%%%%%%%%%%%%%%%%%%%%
\section*{Introduction}
\note{Place holder}

\cite{Bowers_et_al_Phys_Plasmas_2007}

%%%%%%%%%%%%%%%%%%%%%%%%%%%%%%%%%%%%%%%%%%%%%%%%%%%%%%%%%%%%%%%%%%%%%%%%%%%%%%%%
% architecture
%%%%%%%%%%%%%%%%%%%%%%%%%%%%%%%%%%%%%%%%%%%%%%%%%%%%%%%%%%%%%%%%%%%%%%%%%%%%%%%%
\section*{RoadRunner system architecture}

\begin{figure}
    \begin{center}
    \scalebox{0.3}{\input{system.pstex_t}}
    \caption{RoadRunner System Overview}
    \label{fig:system}
    \end{center}
\end{figure}

The RoadRunner supercomputer, shown in figure \ref{fig:system}, is a hybrid, petascale system to be deployed at Los Alamos National Laboratory in 2008.  The system is a first-of-its-kind, heterogeneous cluster-of-clusters that utilizes a combination of 6,912 1.8 GHz, dual-socket, dual-core AMD Opteron \emph{host} processors with 12,960 3.2 GHz, IBM Cell \emph{extended Double-Precision (eDP)} \emph{accelerator} processors.  Each Cell eDP chip is capable of performing 102.4 Gflop/s \textbf{double-precision}, for a total theoretical peak performance of $\sim1.3$ Pflop/s\footnote{The Opteron base system has a theoretical peak performance of $\sim50$ Tflop/s.}.  The RoadRunner supercomputer will be the first machine to achieve a sustained petaflop on the LINPACK benchmark used in ranking the fastest supercomputers in the world for the TOP500 list \cite{top500}.

\subsection*{Connected Unit}

A Connected Unit (CU) is made up of 180 \emph{Triblade}, compute nodes and 12 I/O nodes linked by a first-stage, 288-port Voltaire Infiniband $4x$ DDR switch.  Using a top-down description, the system is comprised of 18 CUs, using eight, second-stage, 288-port Voltaire Infiniband $4x$ DDR switches.  This allows for twelve links per CU to each of the eight switches, with 192 ports \emph{in} and 96 ports \emph{up}, creating a 2-to-1 over-subscribed, fat-tree network topology.

A Connected Unit is a powerful cluster in its own right, with a theoretical peak performance of $\sim74$ Tflop/s.  A single CU of RoadRunner would rank in the top 20 on the current (November 2007) TOP500 list.

\begin{figure}
    \begin{center}
    \scalebox{0.2}{\input{triblade.pstex_t}}
    \caption{Triblade Compute Node}
    \label{fig:triblade}
    \end{center}
\end{figure}

\subsection*{Triblade}

A Triblade compute node, figure \ref{fig:triblade}, actually integrates four physical blades: one IBM LS21, dual-socket Opteron blade, one expansion blade, and two IBM QS22 Cell blades containing the new eDP chips.  The expansion blade connects the two QS22s to the LS21 via four PCI-e $x8$ links and provides the node's ConnectX IB $4x$ DDR link to the rest of the CU cluster.

In aggregate, each Triblade has 32GB RAM, with 16GB on the LS21 blade and 16GB each per QS22 blade, so that each logical processor in a compute node has 4GB RAM.

\note{3 address spaces: Opteron, PPE, SPE LS}
\note{Diskless $\rightarrow$ remote I/O}


%%%%%%%%%%%%%%%%%%%%%%%%%%%%%%%%%%%%%%%%%%%%%%%%%%%%%%%%%%%%%%%%%%%%%%%%%%%%%%%%
% biliography
%%%%%%%%%%%%%%%%%%%%%%%%%%%%%%%%%%%%%%%%%%%%%%%%%%%%%%%%%%%%%%%%%%%%%%%%%%%%%%%%
\bibliographystyle{plain}
\bibliography{bib/gb2008,bib/vpic}

%%%%%%%%%%%%%%%%%%%%%%%%%%%%%%%%%%%%%%%%%%%%%%%%%%%%%%%%%%%%%%%%%%%%%%%%%%%%%%%%
% end document
%%%%%%%%%%%%%%%%%%%%%%%%%%%%%%%%%%%%%%%%%%%%%%%%%%%%%%%%%%%%%%%%%%%%%%%%%%%%%%%%
\end{document}
